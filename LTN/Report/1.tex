% !TEX root = main.tex

\section{Paper Overview}

\subsection{Basic Information}

Here is some basic information about the paper I selected:

\begin{itemize}
    \item Title: Logic Tensor Networks: Deep Learning and Logical Reasoning from Data and Knowledge\cite{serafini2016logic}.
    \item Author: Luciano Serafini and Artur d’Avila Garcez
    \item From: ArXiv.2016
    \item Abstract: This paper proposes Logic Tensor Networks (LTN) to integrate \textbf{learning} and \textbf{reasoning} together based on vector representation. The proposed model LTN represent each object with a vector and then converts each function on multiple objects into a manipulate on their vector representations. Then it also uses a s-norm operator to transform a predicate to a real number in $[0,1]$, which means the confidence of this predicate. Finally, it defines a lose function for all predicates, which means it transfers the reasoning process into a learning and optimization problem.
\end{itemize}

\subsection{Definitions}

Recall that a first-order language $L$ is composed by three parts:
\begin{itemize}
    \item $\mathcal{C}=\{c_1,c_2,\dots\}$, the set of constant symbols;
    \item $\mathcal{F}=\{f_1,f_2,\dots\}$, the set of functional symbols;
    \item $\mathcal{P}=\{p_1,p_2,\dots\}$ the setof predicate symbols.
\end{itemize}


\subsection{Groundings}

LTN defines a term $\mathcal{G}$, called \textbf{grounding}, which contains $\mathcal{G}(c)$, $\mathcal{G}(f)$, $\mathcal{G}(P)$ respect to $c \in \mathcal{C}$, $f\in \mathcal{F}$, $P \in \mathcal{P}$:

Also, LTN define the manipulate of each clause $\phi$.

\subsubsection{Constant}
For each constant, LTN allocate a $n$-dimension vector to it:

\begin{equation}
    \mathcal{G}(c) \in \mathbb{R}^n
\end{equation}

\subsubsection{Function}

For each function $f/m$ on $m$ parameters, LTN define it as a mapping on vector space, which means:

\begin{equation}
    \mathcal{G}(f(t_1,t_2,\dots,t_m)) \in \mathbb{R}^{mn}\rightarrow \mathbb{R}^n
\end{equation}

Note that here we specify the parameter of $f$ is $t_i$ instead of $c_i$ because function $f$ can be recursive like $f(f_1(c1,c2),f_2(c1,c3)$.

\begin{equation}
    \mathcal{G}(f(t_1,t_2,\dots,t_m))=\mathcal{G}(f)\mathcal{G}(t_1),\mathcal{G}(t_2),\dots,\mathcal{G}(t_m)
\end{equation}

More specifically, $\mathcal{G}(f(t_1,t_2,\dots,t_m))$ is fitted by a linear function, which can be implemented with tensor network:

\begin{align}
    \mathcal{G}(f(t_1,t_2,\dots,t_m)) =&\mathcal{G}(f)(v_1,v_2,\dots,v_m) \nonumber\\
    =&\mathcal{G}(f)(v) \nonumber\\
    =&M_f v+N_f
\end{align}

where $v=\left< v_1,v_2,\dots,v_m \right>$, $M_f \in \mathbb{R}^{n\times mn}$, and $N_f \in \mathbb{R}^{n}$

\subsubsection{Predicate}

Like the grounding of a function,

\begin{equation}
    \mathcal{G}(p(t_1,t_2,\dots,t_m)) \in \mathbb{R}^{mn}\rightarrow [0,1]
\end{equation}

Again, as $t_i$ has been mapped to a vector, then:

\begin{equation}
    \mathcal{G}(p(t_1,t_2,\dots,t_m))=\mathcal{G}(P)\mathcal{G}(t_1),\mathcal{G}(t_2),\dots,\mathcal{G}(t_m)
\end{equation}

Therefore we transfer a predicate into a series of manipulation on tensor space:

\begin{align}
    \mathcal{G}(p(t_1,t_2,\dots,t_m))=&\mathcal{G}(f)(v_1,v_2,\dots,v_m) \nonumber \\
    =&\mathcal{G}(f)(v) \nonumber \\
    =&\sigma(u_p^T tanh(v^{T} W_{P} v +V_Pv+B_P))
\end{align}

where $v=\left< v_1,v_2,\dots,v_m \right> \in \mathbb{R}^{mn} $, $W_P \in \mathbb{R}^{mn \times mn \times k}$, $V_P \in \mathbb{R}^{mn \times k}$, $B_P \in \mathbb{R}^{k}$, $u_P \in \mathbb{R}^{k}$, and $\sigma$ is the sigmoid function.

\subsubsection{Clause}

In this project, we assume each clause is disjunctive normal form. So it's easy to get that

\begin{equation}
    \mathcal{G}(\phi_1, \phi_2, \dots, \phi_k)=\mu(\mathcal{G}(\phi_1),\mathcal{G}(\phi_2),\dots,\mathcal{G}(\phi_k))
\end{equation}

where $\mu$ is the max function.

\subsection{Optimization}

So now we get the definition of all components of multi-valued first-order language in tensor networks. The next step is to define a proper lose function.

Saying we know the value of $\phi(x)$ in our dataset, where $x$ is a constant. Intuitively, an optimal grouding should make $\mathcal{G}(\phi(t))$ as close as to $\phi(x)$.
